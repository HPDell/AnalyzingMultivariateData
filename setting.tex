\usepackage{floatrow}
%------------------------------------------------------------------------------
% Ctex
%------------------------------------------------------------------------------
\newcommand{\paragraphnamebox}[1]{\makebox[2em][l]{#1}}
\newcommand{\subparagraphnamebox}[1]{\makebox[2em][l]{#1}}
\setcounter{secnumdepth}{7}
\ctexset{
    section = {
        format = \Large\bfseries\raggedright
    },
    paragraph = {
        name = {(,)},
        number = \arabic{paragraph},
        numbering = true,
        aftername = {},
        nameformat = \paragraphnamebox
    },
    subparagraph = {
        name = {(,)},
        number = \roman{subparagraph},
        numbering = true,
        beforeskip = 0pt,
        aftername = {},
        nameformat = \subparagraphnamebox
    }
}

%------------------------------------------------------------------------------
% Unicode math
%------------------------------------------------------------------------------
\usepackage{amsmath}
\usepackage{amssymb}
\usepackage[T1]{fontenc}
\usepackage[utf8]{inputenc}
\usepackage{mathptmx}
% \usepackage{unicode-math}
% \setmainfont{XITS}
% \setmathfont[bold-style = ISO]{XITS Math}


%------------------------------------------------------------------------------
% Minted
%------------------------------------------------------------------------------
\usepackage{minted}
\setminted{
    autogobble = true,
    linenos=true,
    frame=lines,
    framesep=2mm,
    fontsize=\scriptsize,
    breaklines=true
}
\setminted[py]{
    python3=true,
}
\setmintedinline{
    fontsize=\normalsize
}

%------------------------------------------------------------------------------
% Add new functions
%------------------------------------------------------------------------------
\usepackage{graphicx}
% \usepackage{subfig}
\usepackage{tikz}
\usetikzlibrary{graphs,arrows,shapes,chains,quotes}
\usetikzlibrary{calc}
\usepackage{multirow}
\usepackage{multicol}
\usepackage{booktabs}
\usepackage{tabularx}
\newcolumntype{Y}{>{\centering\arraybackslash}X}
\newcolumntype{Z}{>{\raggedleft\arraybackslash}X}
\usepackage{ltxtable}
\usepackage{makecell}
\usepackage{makeidx}

%------------------------------------------------------------------------------
% Define new macros
%------------------------------------------------------------------------------
\newcommand{\diff}{\mathrm{d }}
\newcommand{\ds}{\displaystyle}
\newcommand{\cpi}{\mathrm{\pi}}
\newcommand{\T}{^\mathrm{T}}
\newcommand{\highlight}[1]{\textcolor{blue}{\textbf{\kaishu{#1}}}}
\newcommand{\done}{\ooalign{$\square$ \cr \raisebox{3pt}{\scriptsize{$√$}}}}
\newcommand{\undone}{\ooalign{$\square$ \cr \raisebox{3pt}{}}}
\newcommand{\cov}[1]{\mathrm{cov}\left({#1}\right)}
\newcommand{\var}[1]{\mathrm{var}\left({#1}\right)}
\newcommand{\corr}[1]{\mathrm{corr}\left({#1}\right)}
\newcommand{\ddfrac}[2]{\displaystyle{\frac{\displaystyle{#1}}{\displaystyle{#2}}}}
\newlength{\subparitemindent}
\newlength{\paritemindent}
\setlength{\paritemindent}{3em}
\setlength{\subparitemindent}{5em}
\newenvironment{paritemize}{
    \begin{itemize}[leftmargin=\paritemindent]
}{
    \end{itemize}
    \hangafter=1\setlength{\hangindent}{2em}
}
\newenvironment{subparitemize}{
    \begin{itemize}[leftmargin=\subparitemindent]
}{
    \end{itemize}
    \hangafter=1\setlength{\hangindent}{4em}
}
\newcommand{\hparagraph}[1]{\paragraph{#1} \hangafter=1\setlength{\hangindent}{2em}}
\newcommand{\hsubparagraph}[1]{\subparagraph{#1} \hangafter=1\setlength{\hangindent}{4em}}
\newcommand{\hsubparagraphcontinue}{\setlength{\leftskip}{3em}\hangafter=1\setlength{\hangindent}{1em}}
\newcommand{\hfrac}[2]{\left. {#1} \middle/ {#2} \right.}
\newcommand{\mindex}[2]{{\kaishu{#1}}(#2)\index{{#1}({#2})}}
\newcommand{\rsquare}{$ R^2 $}
\newcommand{\diag}{\mathrm{diag}}
%------------------------------------------------------------------------------
% Format setting
%------------------------------------------------------------------------------
\usepackage[left=1.91cm, right=1.91cm, top=1.91cm, bottom=1.91cm]{geometry}
\hypersetup{colorlinks=true,bookmarksnumbered=true}
% \allowdisplaybreaks
\numberwithin{equation}{chapter}
% \numberwithin{figure}{section}
\usepackage[inline]{enumitem}
\setlist{topsep=0pt,partopsep=0pt,parsep=0pt,itemsep=0pt,itemjoin=\quad\quad,leftmargin=4em,labelwidth=2em,labelsep=0em,align=left}
%% 脚注每页重新标号
\usepackage[perpage]{footmisc}
% \setcounter{secnumdepth}{3}
% 积分号样式
\usepackage[fourier]{varint}
% 表格行距
% \renewcommand\arraystretch{1.3}
% 题注格式
\usepackage{caption}
\captionsetup{labelfont=bf,labelsep=quad}
\DeclareCaptionSubType[alph]{figure} 
\captionsetup[subfigure]{labelformat=parens,labelsep=space,justification=centerlast}
% Autoref 格式
\renewcommand{\eqref}[1]{(\ref{#1})}